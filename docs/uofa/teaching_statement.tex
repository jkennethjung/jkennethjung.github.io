\documentclass[11pt,letterpaper]{article}
\usepackage[margin=0.9in]{geometry}
\usepackage{enumitem}
\usepackage{parskip}
\usepackage{titlesec}
\usepackage{url}
\usepackage{xcolor}
\usepackage{microtype}
\usepackage{helvet}

% Set default font to Helvetica (sans-serif)
\renewcommand{\familydefault}{\sfdefault}

% Remove page numbering
\pagestyle{empty}

% Define colors
\definecolor{mediumgray}{RGB}{102,102,102}
\definecolor{lightgray}{RGB}{128,128,128}

% Customize section formatting
\titleformat{\section}
  {\Large\bfseries}
  {\thesection}{0em}{}
\renewcommand{\titlerule}{\color{black}\hrule width\textwidth height 0.8pt}
\titlespacing*{\section}{0pt}{12pt}{6pt}

% Customize subsection formatting
\titleformat{\subsection}
  {\large\bfseries}
  {\thesubsection}{0em}{}
\titlespacing*{\subsection}{0pt}{8pt}{4pt}

% Custom command for job titles/positions
\newcommand{\jobtitle}[1]{\textbf{\color{mediumgray}#1}}
\newcommand{\institution}[1]{\textit{#1}}
\newcommand{\daterange}[1]{\textcolor{lightgray}{#1}}

% Remove list indentation and customize
\setlist[itemize]{leftmargin=0pt, itemsep=2pt, parsep=0pt}

% Custom header environment
\newenvironment{cvheader}
{\begin{center}\Large}
{\end{center}}

% Custom contact info environment
\newenvironment{contactinfo}
{\small\begin{tabular*}{\textwidth}{@{\extracolsep{\fill}}ll}}
{\end{tabular*}}

\begin{document}

%% Header with name
\begin{cvheader}
{\Huge\textbf{Teaching Statement on Inclusion}}\\
{\large\textcolor{mediumgray}{J. Kenneth Jung}}
\end{cvheader}

\vspace{8pt}

\section*{Teaching Philosophy}

My teaching philosophy emphasizes meeting students where they are, both academically and personally. I have worked extensively with students facing various challenges, from those struggling with mental health issues—whom I helped connect with campus resources—to students encountering coding difficulties, with whom I spent hours troubleshooting scripts line by line. When developing teaching materials for environmental economics, I prioritized accessibility by creating clear explanations of concepts ranging from competitive equilibria to environmental justice, supplemented with real-world examples and recent research applications. All materials were made available as PDFs online to ensure broad accessibility. This approach reflects my belief that academic excellence and inclusivity are not competing goals but rather mutually reinforcing elements of effective education. In my future role, I will continue to prioritize accessible pedagogy, proactive student support, and the creation of learning environments where all students can thrive regardless of their starting point or background.

\section*{Creating Inclusive Learning Environments}

My approach to fostering equity and belonging in teaching centers on creating inclusive learning environments that recognize and respond to students' diverse backgrounds and needs. Through my experience as a Teaching Fellow across environmental economics, econometrics, and other courses, I have learned that effective teaching requires moving beyond one-size-fits-all approaches. For example, when redesigning the structure of the Senior Essay course for those seeking to write a bachelor's thesis in economics, I changed the traditional appointment-based office hours into accessible walk-in sessions after observing that the existing system was underutilized. This simple change dramatically increased attendance and created a more welcoming space where students could drop in to work individually or seek help as needed. I also established regular roundtable meetings where students could present their work formally or discuss progress informally, providing structure while maintaining flexibility. These innovations emerged from listening to student needs through surveys and direct feedback, demonstrating my commitment to responsive pedagogy that adapts to serve all learners effectively.

\section*{Building Collaborative Academic Communities}

Similarly, in approaching research I believe that breaking down artificial barriers is essential for fostering collaborative relationships across academic communities and advancing work that holds meaning to broader audiences. For example, at my home institution I noticed a somewhat artificial separation between the Department of Economics and the economics group at the Yale School of the Environment (YSE) in spite of their shared research interests. During my time as a PhD candidate, I actively participated in the environmental economics seminars at YSE and befriended many in the community. These efforts led to many fruitful discussions about research with students and faculty, presentation of my own research at the YSE Forest School, as well as coauthorship with an environmental economics PhD student. This experience taught me the importance of creating bridges between communities and the value of cross-disciplinary collaboration. I also co-founded Industrial Organization Tea, a forum for economics PhD students to share their work in a supportive and collegial setting. Just as in my coordination of roundtable meetings for the Senior Essay, my organization of IO Tea shaped my understanding of how intentional community-building efforts can cultivate belonging across diverse academic spaces.

\end{document}