\documentclass[11pt,letterpaper]{article}
\usepackage[margin=0.9in]{geometry}
\usepackage{enumitem}
\usepackage{parskip}
\usepackage{titlesec}
\usepackage{url}
\usepackage{xcolor}
\usepackage{microtype}
\usepackage{helvet}

% Set default font to Helvetica (sans-serif)
\renewcommand{\familydefault}{\sfdefault}

% Remove page numbering
\pagestyle{empty}

% Define colors
\definecolor{mediumgray}{RGB}{102,102,102}
\definecolor{lightgray}{RGB}{128,128,128}

% Customize section formatting
\titleformat{\section}
  {\Large\bfseries}
  {\thesection}{0em}{}
\renewcommand{\titlerule}{\color{black}\hrule width\textwidth height 0.8pt}
\titlespacing*{\section}{0pt}{12pt}{6pt}

% Customize subsection formatting
\titleformat{\subsection}
  {\large\bfseries}
  {\thesubsection}{0em}{}
\titlespacing*{\subsection}{0pt}{8pt}{4pt}

% Custom command for job titles/positions
\newcommand{\jobtitle}[1]{\textbf{\color{mediumgray}#1}}
\newcommand{\institution}[1]{\textit{#1}}
\newcommand{\daterange}[1]{\textcolor{lightgray}{#1}}

% Remove list indentation and customize
\setlist[itemize]{leftmargin=0pt, itemsep=2pt, parsep=0pt}

% Custom header environment
\newenvironment{cvheader}
{\begin{center}\Large}
{\end{center}}

% Custom contact info environment
\newenvironment{contactinfo}
{\small\begin{tabular*}{\textwidth}{@{\extracolsep{\fill}}ll}}
{\end{tabular*}}

\begin{document}

%% Header with name
\begin{cvheader}
{\Huge\textbf{Research Statement}}\\
{\large\textcolor{mediumgray}{J. Kenneth Jung}}
\end{cvheader}

\vspace{8pt}

My research agenda applies industrial organization (IO) methods to pressing questions in environmental and resource economics. This intersection proves particularly fruitful because many environmental policy questions inherently involve strategic interactions between firms, government agencies, and other stakeholders—precisely the types of complex market dynamics that industrial organization is designed to analyze. My work demonstrates how tools traditionally used to study competition and market structure can shed light on critical environmental policy questions that require understanding both the underlying resource dynamics and the strategic behavior of economic agents.

My job market paper exemplifies this approach by examining how climate change-driven environmental shocks create urgent policy design challenges. I study the mountain pine beetle outbreak that threatened western North American forests, focusing on how policymakers designed timber auctions to encourage rapid harvesting while maximizing revenue. The key insight is that effective policy design required understanding how logging firms' harvesting decisions depend on resource quality, how they form beliefs about beetle spread patterns, and how they compete with each other for harvesting rights. By modeling these strategic interactions through structural methods, I am able to measure the full effects of different auction designs. Similarly, my coauthored work on additionality in fishery buybacks addresses the question of additionality in environmental economics: how can policymakers determine whether paying someone to change their behavior actually did so? We identify a novel channel where buyback programs increase the value of remaining licenses by reducing competition, and find preliminary evidence of this mechanism in Texas shrimp fisheries. Our ongoing work involves modeling shrimping behavior as a function of buyback opportunities and market conditions to answer whether gradual versus immediate fund disbursement would improve program effectiveness. This, again, is a policy counterfactual that requires structural modeling techniques central to industrial organization.

I am attaching my job market paper draft, which is currently incomplete and contains just reduced form results so far. The project is progressing rapidly, as I am currently computing policy counterfactuals using estimates from my structural model. These counterfactual simulations are time-consuming to run and test, with the first round of results having just finished recently. I am happy to append the next set of results to the paper in the coming week or two upon request.

\end{document}