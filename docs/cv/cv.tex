\documentclass[11pt,letterpaper]{article}
\usepackage[margin=0.9in]{geometry}
\usepackage{enumitem}
\usepackage{parskip}
\usepackage{titlesec}
\usepackage{url}
\usepackage{xcolor}
\usepackage{microtype}
\usepackage{helvet}

% Set default font to Helvetica (sans-serif)
\renewcommand{\familydefault}{\sfdefault}

% Remove page numbering
\pagestyle{empty}

% Define colors
\definecolor{mediumgray}{RGB}{102,102,102}
\definecolor{lightgray}{RGB}{128,128,128}

% Customize section formatting
\titleformat{\section}
  {\Large\bfseries}
  {\thesection}{0em}{}[\titlerule[0.8pt]]
\titlespacing*{\section}{0pt}{12pt}{6pt}

% Customize subsection formatting
\titleformat{\subsection}
  {\large\bfseries}
  {\thesubsection}{0em}{}
\titlespacing*{\subsection}{0pt}{8pt}{4pt}

% Custom command for job titles/positions
\newcommand{\jobtitle}[1]{\textbf{\color{mediumgray}#1}}
\newcommand{\institution}[1]{\textit{#1}}
\newcommand{\daterange}[1]{\textcolor{lightgray}{#1}}

% Remove list indentation and customize
\setlist[itemize]{leftmargin=0pt, itemsep=2pt, parsep=0pt}

% Custom header environment
\newenvironment{cvheader}
{\begin{center}\Large}
{\end{center}}

% Custom contact info environment
\newenvironment{contactinfo}
{\small\begin{tabular*}{\textwidth}{@{\extracolsep{\fill}}ll}}
{\end{tabular*}}

\begin{document}

% Header with name
\begin{cvheader}
{\Huge\textbf{J. Kenneth Jung}}\\
{\large\textcolor{mediumgray}{Ph.D. Candidate in Economics}}
\end{cvheader}

\vspace{8pt}

% Contact information in two columns
\begin{center}
\begin{tabular}{ll}
\textbf{Address} & \textbf{Contact} \\
30 Hillhouse Ave & \textbf{Phone:} +1-(336)-745-0714 \\
Department of Economics & \textbf{Email:} ken.jung@yale.edu \\
Yale University & \textbf{Web:} jkennethjung.github.io \\
New Haven, CT 06520-8268 & \textbf{Citizenship:} United States \\
\end{tabular}
\end{center}

\vspace{8pt}

\section*{Research Interests}
\textbf{Primary Fields:} Industrial Organization, Environmental and Resource Economics

\section*{Education}

\textbf{Yale University}, New Haven, CT \\
\jobtitle{Ph.D., Economics} \hfill \daterange{2025 (expected)} \\
\jobtitle{M.Phil., Economics} \hfill \daterange{2022} \\
\jobtitle{M.A., Economics} \hfill \daterange{2021} \\

\textbf{University of Chicago}, Chicago, IL \\
\jobtitle{B.A., Economics} \hfill \daterange{2017}

\section*{Dissertation}

\textbf{Title:} Essays in Industrial Organization and Resource Economics

\section*{Academic Progress}

\textbf{Comprehensive Examinations:}
\begin{itemize}
\item \textbf{2021 (Oral):} Industrial Organization, Political Economy
\item \textbf{2020 (Written):} Microeconomics, Macroeconomics  
\end{itemize}

\section*{Research}

\subsection*{Job Market Paper}
``Moral Hazard in Resource Extraction: Evidence from the Mountain Pine Beetle Outbreak''

\subsection*{Work in Progress}
\begin{itemize}
\item ``Additionality and Leakage in Equilibrium'' \textit{with Andrew Vogt}
\item ``Aircraft Leakage under Cap and Trade'' \textit{with Meichen Chen and Miho Hong}
\end{itemize}

\newpage

\section*{Teaching Experience}

\textbf{Yale College}

\jobtitle{Teaching Assistant} \hfill \daterange{Spring 2025} \\
\institution{Introduction to Data Science and Econometrics} (Prof. John Eric Humphries)

\jobtitle{Teaching Assistant} \hfill \daterange{Fall 2023--Spring 2024} \\
\institution{The Senior Essay} (Prof. Rebecca Toseland)

\jobtitle{Teaching Assistant} \hfill \daterange{Spring 2023} \\
\institution{Intermediate Econometrics} (Prof. Edward Vytlacil)

\jobtitle{Teaching Assistant} \hfill \daterange{Fall 2022} \\
\institution{Environmental Economics} (Prof. Robert Mendelsohn)

\jobtitle{Teaching Assistant} \hfill \daterange{Spring 2022} \\
\institution{Industrial Organization} (Prof. Philip Haile)

\jobtitle{Teaching Assistant} \hfill \daterange{Fall 2021} \\
\institution{Intermediate Microeconomics} (Prof. Evangelia Chalioti)

\section*{Research Experience}

\jobtitle{Research Assistant} \hfill \daterange{Summer 2023} \\
\institution{Yale University} (Prof. Nicholas Ryan)

\jobtitle{Research Assistant} \hfill \daterange{2017--2019} \\
\institution{Massachusetts Institute of Technology} (Prof. Amy Finkelstein)

\section*{Presentations}

\textbf{Conference and Seminar Presentations:}
\begin{itemize}
\item London School of Economics and Political Science, Environment Camp \hfill \daterange{2025}
\item University of Colorado at Boulder, Environmental and Resource Economics Workshop \hfill \daterange{2023}
\end{itemize}

\section*{Professional Service}

\textbf{Referee Service:} American Economic Review

\section*{Additional Information}

\textbf{Languages:} English (native), Korean (intermediate), French (beginner)

\newpage

\section*{References}

\begin{minipage}[t]{0.48\textwidth}
\textbf{Prof. Katja Seim} \\
Yale University \\
Department of Economics \\
New Haven, CT 06520 \\
\textbf{Phone:} 203-432-5487 \\
\textbf{Email:} katja.seim@yale.edu
\end{minipage}
\hfill
\begin{minipage}[t]{0.48\textwidth}
\textbf{Prof. Philip Haile} \\
Yale University \\
Department of Economics \\
New Haven, CT 06520 \\
\textbf{Phone:} 203-432-3568 \\
\textbf{Email:} philip.haile@yale.edu
\end{minipage}

\vspace{12pt}

\begin{minipage}[t]{0.48\textwidth}
\textbf{Prof. Kenneth Gillingham} \\
Yale University \\
School of the Environment \\
New Haven, CT 06520 \\
\textbf{Phone:} 203-436-5465 \\
\textbf{Email:} kenneth.gillingham@yale.edu
\end{minipage}
\hfill
\begin{minipage}[t]{0.48\textwidth}
\textbf{Prof. Steven Berry} \\
Yale University \\
Department of Economics \\
New Haven, CT 06520 \\
\textbf{Phone:} 203-432-3556 \\
\textbf{Email:} steven.berry@yale.edu
\end{minipage}

\newpage

\section*{Dissertation Abstract}

\subsection*{Moral Hazard in Resource Extraction: Evidence from the Mountain Pine Beetle Outbreak [Job Market Paper]}

Natural resource owners often aim to advance a range of conservation or management goals while raising revenues, but contract out extraction rights to firms. A key factor for many such objectives is the rate of extraction, particularly when the quality of the resource is dynamic or when extraction prevents losses due to fires or disease. I study a setting with both features that is of growing concern under a warming climate: timber auctions for bark beetle-infested forests. In particular, I show that the use of a contract that underpriced low-grade logs made it profitable for firms to delay harvests, allowing pupating beetles to spread.

The empirical setting is the 1999-2015 mountain pine beetle outbreak in British Columbia. Traditionally, timber auctions sold by the province follow a scaled format in which bidders submit the price they will pay per unit of wood extracted. However, this price only applies to logs that are largely free of defects, whereas a small, fixed fee of \$0.25 per cubic meter is charged for all other logs. In contrast, the low-grade fee is absent in lump sum auctions, in which the bid is a single payment for the entire harvest. A regression discontinuity for bidding formats reveals that harvests in the marginal scaled auction are 2.5 months more delayed than in comparable lump sum auctions. However, unit pricing absent the low-grade fee would still cause delays because it raises the prices at which firms are willing to extract, so the estimate bundles both effects.

To disentangle distortions caused by unit pricing versus low-grade fees, I tie bidder valuations at the auction stage to a dynamic extraction problem modeling all channels by which market conditions and tree quality affect revenues and harvesting costs. With the time horizon given by the contract's term length, the problem of how much of the remaining timber stock to harvest each month is solved backwards. I leverage reports that quantify beetle damage just prior to the auction, so that tree quality is an initially observed state that evolves stochastically. In both auction formats, the winning firm forfeits the option value of better lumber prices for the revenue of selling higher-grade trees by harvesting earlier; however, in the scaled auction, the firm reduces harvesting payments by selling more logs of lower quality. Anticipating this dynamic problem, firms at the time of the auction bid based on its expected value as well as their private costs.

The model estimates reveal that the low-grade fee is highly valued by scaled auction bidders, but that most of the surplus it provides is competed away in equilibrium, resulting in little revenue loss. Using aerial data on beetle-caused tree mortality, I find that moral hazard led to spillovers, even after accounting for the beetle's seasonal mating cycle. Adjusting contract terms to avoid spillovers would have been very costly because bidders lose significant option value under shorter contracts; however, lump sum auctions would have achieved half of the reduction in delay at a third of the cost.

\end{document}
